\usepackage[UTF8]{ctex}
\usepackage{mathptmx}
\usepackage{amsmath, amsthm, amssymb} %数学符号等相关宏包
\usepackage[utf8]{inputenc}
\usepackage{graphicx} %插入图片所需宏包
\usepackage{caption} %实现图片的多行说明
\usepackage{xspace} %提供一些好用的空格命令
\usepackage{url} %更好的超链接显示
\usepackage{float} %图片与表格的更好排版
\usepackage{ulem} %更好的下划线
\usepackage{setspace}% 行距等
\usepackage[a4paper,top=2.0cm, bottom=2.0cm, left=3.0cm, right=3.0cm]{geometry} %设置页边距
\usepackage{booktabs}
\usepackage{indentfirst}
\usepackage{titlesec}
\usepackage{fancyhdr}
\usepackage{makecell} %表格内换行
\usepackage{svg}
\usepackage{array}
\usepackage{abstract}
\usepackage{calc}
\usepackage{titletoc} % 设置目录格式
\usepackage{emptypage}
\usepackage{atbegshi}

\usepackage{times}
\usepackage{xcolor} %彩色的文字
\usepackage[bookmarks=true, colorlinks, citecolor=blue, linkcolor=black]{hyperref} %超链接包
\usepackage{cleveref} %交叉引用
\usepackage{listings}%code highlighting

%使用biblatex管理文献,输出格式使用gb7714-2015标准,后端为biber
\usepackage[backend=biber,style=gb7714-2015,hyperref=true]{biblatex}



%允许公式跨页显示
\allowdisplaybreaks
\setstretch{1.5}

\setlength{\tabcolsep}{1mm}{}

\numberwithin{figure}{section}
\numberwithin{table}{section}

\renewcommand{\abstractnamefont}{\heiti\zihao{4}}
\renewcommand{\abstracttextfont}{\zihao{-4}}
\setlength{\absleftindent}{0pt}
\setlength{\absrightindent}{0pt}


\titleformat
{\section}
{\centering\songti\zihao{3}}
{第 \thesection 章}
{1em}
{}

\titleformat
{\subsection}
{\songti\zihao{4}}
{\thesubsection}
{1em}
{}

\titleformat
{\subsubsection}
{\songti\zihao{-4}}
{\thesubsubsection}
{1em}
{}

\newcommand{\jluTitle}[1]{
\begin{center}
\songti \zihao{-3} #1
\end{center}
}

\newcounter{contentpage}
\setcounter{contentpage}{1}

\AtBeginShipout{%
  \addtocounter{contentpage}{1}
}

\pagestyle{fancy}
\fancyhf{}
\fancyhead[L]{}
\fancyhead[R]{}
\fancyhead[C]{吉林大学\quad{}软件学院\quad{}毕业论文}
\renewcommand{\headrulewidth}{1pt}%分隔线宽度1磅
\fancyfoot[OR]{\bfseries ~\thecontentpage~}
\fancyfoot[EL]{\bfseries ~\thecontentpage~}

\newcommand{\alignOddPage}{
    % \cleardoublepage
    \ifodd\value{page}
        
    \else
        \thispagestyle{empty}
        \null
        \newpage
        \addtocounter{contentpage}{-1}
    \fi
}

\newcommand{\RomanNumbering}{
    \fancyfoot[OR]{\bfseries ~\Roman{contentpage}~}
    \fancyfoot[EL]{\bfseries ~\Roman{contentpage}~}
}

\newcommand{\arabicNumbering}{
    \fancyfoot[OR]{\bfseries ~\arabic{contentpage}~}
    \fancyfoot[EL]{\bfseries ~\arabic{contentpage}~}
}

\newcolumntype{L}[1]{>{\raggedright\let\newline\\\arraybackslash\hspace{0pt}}m{#1}}
\newcolumntype{C}[1]{>{\centering\let\newline\\\arraybackslash\hspace{0pt}}m{#1}}
\newcolumntype{R}[1]{>{\raggedleft\let\newline\\\arraybackslash\hspace{0pt}}m{#1}}

%调整目录格式
\titlecontents{section}
[0pt]
{\addvspace{1.5pt}\filright\bf}
{\contentspush{第\thecontentslabel\ 章\quad}}
{}
{\titlerule*[10pt]{.}\contentspage}

\titlecontents{subsection}
[1em] % 左侧间距,增大这个值会加大subsection的缩进
{\addvspace{0.5pt}} % above code
{\contentspush{\thecontentslabel\ }} % numbered entry format
{} % numberless entry format
{\titlerule*[10pt]{.}\contentspage} % filler page format

\titlecontents{subsubsection}
[2em] % 左侧间距,增大这个值会加大subsubsection的缩进
{\addvspace{0.5pt}} % above code
{\contentspush{\thecontentslabel\ }} % numbered entry format
{} % numberless entry format
{\titlerule*[10pt]{.}\contentspage} % filler page format