{

\zihao{-3}
\makeatletter
\renewcommand{\arraystretch}{1.2}

% inner temp length
\newlength\@tempTitleHt
\newlength\@tempTitleMaxWd

% length used as parameters, "Wd" is a shorthand for "Width"
\newlength\titleSingleLineMaxWd
\newlength\titleMultiLineMaxWd
\newlength\titleLRExtraWd
\newlength\titleSepWd

\setlength\titleSingleLineMaxWd{0.6\textwidth}
\setlength\titleMultiLineMaxWd{0.5\textwidth}
\setlength\titleLRExtraWd{.5em}
\setlength\titleSepWd{.5em}

% draw a fixed-width underline
\newcommand{\titleUnderline}{%
  \rule[-.3em]{\titleSingleLineMaxWd + 2\titleLRExtraWd}{.5pt}}

% title content typesetter
\newcommand\titleBox[1]{%
  \setlength\@tempTitleMaxWd\titleSingleLineMaxWd
  % Measure the total height of #1 subject to \titleSingleLineMaxWd.
  % Here \@titleBox is necessary when #1 contains "\\".
  \settototalheight\@tempTitleHt{\@titleBox{#1}}%
  \ifdim\@tempTitleHt=0pt%
    % case 1: #1 causes empty output
    \titleUnderline
  \else
    % Change to LR mode, for inserting a zero-width box right after.
    \leavevmode
    % note: Use \normalbaseineskip instead of \baselineskip, 
    %       since the latter is set to 0pt inside tabular env.
    \ifdim\@tempTitleHt>\normalbaselineskip
      % case 2: #1 is fit for multilines, or contains `\\`, hence
      %         \titleMultiLineMaxWd is used instead, and the total 
      %         height of #1 subject to new max width is re-measured.
      \setlength\@tempTitleMaxWd\titleMultiLineMaxWd
      \settototalheight\@tempTitleHt{\@titleBox{#1}}%
      % \rlap{\smash{...}} typesets content of its argument normally 
      % but sets it into a zero sized box. Here \@titleBox set 
      % (possiblly) multi-line text into a single box, since \smash 
      % only takes in argument spreading one line.
      %
      % Every line of title content needs an underline, hense we do
      % a loop to typeset one underline for every line the title 
      % content actually spreads.
      \rlap{\smash{\@titleBox{%
        \@whiledim\@tempTitleHt>0pt%
        \leavevmode
        % 
        \do{%
          \rlap{\titleUnderline}\\%
          \addtolength\@tempTitleHt{-\normalbaselineskip}%
        }%
      }}}%
      % Insert h-offset to manually center the title content.
      \hspace{\dimexpr\titleLRExtraWd + .5\titleSingleLineMaxWd - .5\titleMultiLineMaxWd\relax}%
      \@titleBox{\centering #1}%
    \else
      % case 3: #1 is fit for one line
      \rlap{\titleUnderline}%
      \hspace{\titleLRExtraWd}%
      \@titleBox{\centering #1}%
    \fi
  \fi
}

% common operations on every title content
\newcommand{\@titleBox}[1]{%
  \parbox[t]{\@tempTitleMaxWd}{#1}}
\makeatother
\newlength{\lwtm}
\setlength{\lwtm}{\widthof{ 论文题目 }}

\newpage
\thispagestyle{empty}

~\\

\begin{figure}[h]
    \centering
    \includegraphics[scale=0.4]{article/figure/cover.png}
\end{figure}

{
\begin{center}
    \zihao{1}\textbf{本科生毕业论文(设计)}
\end{center}
}

~\\

\begin{tabular}{p{\lwtm} p{\titleSingleLineMaxWd}}
    \makebox[\lwtm][s]{中文题目}           & \titleBox{HELLO WORLD} \\
    \makebox[\lwtm][s]{英文题目}           & \titleBox{HELLO WORLD} \\
    \makebox[\lwtm][s]{学生姓名}           & \titleBox{OceanPresent} \\
    \makebox[\lwtm][s]{学号}           & \titleBox{HELLO WORLD} \\
                                            % &               \\[-10pt] 
                                            % 用于分隔题目与其他信息
    \makebox[\lwtm][s]{学院}           &    \titleBox{HELLO WORLD}          \\
    \makebox[\lwtm][s]{专业}               &   \titleBox{HELLO WORLD}\\
    \makebox[\lwtm][s]{指导教师}               &   \titleBox{HELLO WORLD} \\
\end{tabular}

~\\

\begin{center}
    {\zihao{3}\textbf{2024 年 6 月}}
\end{center}
}